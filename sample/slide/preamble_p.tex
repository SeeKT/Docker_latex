% パッケージなど
\usepackage{bxdpx-beamer}
%\usepackage{pxjahyper}
%\usepackage{minijs}
%\usepackage{algorithm, algorithmic}
\usepackage{graphicx, url, float, booktabs, listings, color, pdfpages, amsmath, amssymb, latexsym, mathtools, ascmac, amsfonts, comment, bm}
\usepackage{color}
%\usepackage{diagbox}
%\usepackage{otf}
\usepackage{multirow,array}
% textbfのフォント設定
%\renewcommand{\textbf}[1]{{\bfseries\sffamily#1}}

% 定理環境
%\newtheorem{definition}{Definition}

%%%%%%%%%% Meiryoを使いたい %%%%%%%%%%
% 参考: https://risa.is.tokushima-u.ac.jp/~tetsushi/howtomakeslides.pdf
%\usepackage{zxjatype}
%\setCJKmainfont[Scale=0.95]{Meiryo}
% Ubuntuの場合,Windows PCからttcファイルをコピーする (参考: https://the-bluestars.com/2020/07/21/47/)
% コンパイルはxelatexを使う
%%%%%%%%%%%%%%%%%%%%%%%%%%%%%%%%%%%%%

% 下にあるナビゲーションシンボルを消す
\setbeamertemplate{navigation symbols}{}
% 2020/11/6
% itemizeが表示されない問題に対して,これを追加
% https://oku.edu.mie-u.ac.jp/tex/mod/forum/discuss.php?d=2963
\makeatletter
\def\pgfutil@insertatbegincurrentpagefrombox#1{%
  \edef\pgf@temp{\the\wd\pgfutil@abb}%
  \global\setbox\pgfutil@abb\hbox{%
    \unhbox\pgfutil@abb%
    \hskip\dimexpr2in-2\hoffset-\pgf@temp\relax% changed
    #1%
    \hskip\dimexpr-2in-2\hoffset\relax% new
  }%
}
\makeatother

% textposパッケージによって,絶対座標で図表位置を指定
% https://blog.tokor.org/2018/01/19/LaTeX-Beamer%E3%81%A7%E5%9B%B3%E8%A1%A8%E3%81%AA%E3%81%A9%E3%81%AE%E4%BD%8D%E7%BD%AE%E3%82%92%E7%B5%B6%E5%AF%BE%E5%BA%A7%E6%A8%99%E3%81%A7%E6%8C%87%E5%AE%9A%E3%81%99%E3%82%8B%E6%96%B9%E6%B3%95/
\usepackage[absolute,overlay]{textpos}

% デフォルトのフォントをゴシックにする
%\renewcommand{\kanjifamilydefault}{\gtdefault}

% デフォルトのフォントをArialにする
%\renewcommand{\rmdefault}{phv} % Arial
%\renewcommand{\sfdefault}{phv} % Arial


% 数式フォントをarticleと同じにする
\usefonttheme{professionalfonts}

% 丸数字 (title)
% https://livingdead0812.hatenablog.com/entry/20161005/1475654232
\newcommand{\ctext}[1]{\raise0.2ex\hbox{\textcircled{\large{#1}}}}


%% 各sectionごとに目次スライドの表示
\AtBeginSection[]{
    \frame{\tableofcontents[currentsection]} %目次スライド
}

% 色の定義: https://irocore.com/
% rgbの値を256で割る
% 赤系
\definecolor{shinshu}{rgb}{0.83984375, 0.14453125, 0} % 真朱
\definecolor{shinku}{rgb}{0.67578125, 0, 0.17578125}  % 深紅
\definecolor{akadaidai}{rgb}{0.9140625, 0.33203125, 0.0234375}  % 赤橙
% 青系
\definecolor{konpeki}{rgb}{0, 0.35546875, 0.59375}  % 紺碧
\definecolor{seiran}{rgb}{0, 0.33203125, 0.55859375}  % 青藍
\definecolor{amairo}{rgb}{0, 0.5234375, 0.796875} % 天色
\definecolor{kon}{rgb}{0, 0.1015625, 0.26171875}  % 紺
\definecolor{gunjou}{rgb}{0, 0.35546875, 0.6640625} % 群青色
% 緑系
\definecolor{shinpeki}{rgb}{0, 0.3671875, 0.08203125} % 深碧
% 黄系
\definecolor{lemon}{rgb}{0.921875, 0.87109375, 0.16796875}  % 檸檬色
\definecolor{yamabuki}{rgb}{0.96875, 0.703125, 0} % 山吹色
% アンダーライン
\newcommand{\redunderline}[1]{\textcolor{akadaidai}{\underline{\textcolor{black}{#1}}}}   %赤
\newcommand{\blueunderline}[1]{\textcolor{gunjou}{\underline{\textcolor{black}{#1}}}}   %青
\newcommand{\greenunderline}[1]{\textcolor{shinpeki}{\underline{\textcolor{black}{#1}}}}   %青
\newcommand{\yellowunderline}[1]{\textcolor{yamabuki}{\underline{\textcolor{black}{#1}}}}   %青

% 補足スライド (ページ番号のmaxを更新しない)
\newcommand{\backupbegin}{
   \newcounter{framenumberappendix}
   \setcounter{framenumberappendix}{\value{framenumber}}
}
\newcommand{\backupend}{
   \addtocounter{framenumberappendix}{-\value{framenumber}}
   \addtocounter{framenumber}{\value{framenumberappendix}}
}

\usepackage{tcolorbox}
% These options will be applied to all `tcolorboxes`
\tcbset{%
  noparskip,
  colback=gray!10, %background color of the box
  colframe=gray!40, %color of frame and title background
  coltext=black, %color of body text
  coltitle=black, %color of title text
  fonttitle=\bfseries,
  left=4pt, right=4pt,
  alerted/.style={coltitle=white,
                  colframe=red!75!black,
                  colback=red!5!white},
  example/.style={coltitle=black,
                   colframe=green!20,
                   colback=green!5},
}

% tex 文字色
\newcommand{\red}[1]{\textcolor{akadaidai}{#1}} % 赤 (赤橙)
\newcommand{\rred}[1]{\textcolor{red}{#1}}
\newcommand{\bblue}[1]{\textcolor{blue}{#1}}
\newcommand{\blue}[1]{\textcolor{gunjou}{#1}} % 青 (群青色)
\newcommand{\green}[1]{\textcolor{shinpeki}{#1}} % 緑 (深碧)
\newcommand{\yellow}[1]{\textcolor{yamabuki}{#1}}  % 黄 (檸檬)

% caption
\usepackage[hang,small,bf]{caption}
\usepackage[subrefformat=parens]{subcaption}

% beamerのテーマ
%\usetheme{default}
%\usetheme{AnnArbor}
%\usetheme{Antibes}
%\usetheme{Bergen}
%\usetheme{Berkeley}
%\usetheme{Berlin}
\usetheme{Boadilla}
%\usetheme{CambridgeUS}
%\usetheme{Copenhagen}
%\usetheme{Darmstadt}
%\usetheme{Dresden}
%\usetheme{Frankfurt}
%\usetheme{Goettingen}
%\usetheme{Hannover}
%\usetheme{Ilmenau}
%\usetheme{JuanLesPins}
%\usetheme{Luebeck}
%\usetheme{Madrid}
%\usetheme{Malmoe}
%\usetheme{Marburg}
%\usetheme{Montpellier}
%\usetheme{PaloAlto}
%\usetheme{Pittsburgh}
%\usetheme{Rochester}
%\usetheme{Singapore}
%\usetheme{Szeged}
%\usetheme{Warsaw}
